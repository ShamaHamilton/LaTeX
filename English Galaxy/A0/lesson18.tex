\subsection{Урок №18}

\subsection*{\centering{Словарь}}
\begin{multicols}{2}
    \begin{enumerate}\setlength{\itemsep}{0pt}
        \item text - текст
        \item writer - писатель
        \item sofa - диван
        \item under - под
        \item smile - улыбаться, улыбка
        \item simple - простой
        \item necessary - необходимо
        \item photo - фотография
        \item moment - момент
        \item each - каждый
        \item lie - лежать
        \item relax - отдыхать, расслабляться
        \item tradition - традиция
    \end{enumerate}
\end{multicols}

\subsection*{\centering{Теория}}
Помимо артиклей \underline{a} и \underline{an} в английском языке существует еще артикль \underline{the}.
Он называется определенным и ставится перед существительными, обозначающими конкретный предмет, человека,
явление и т.д. Так же полезно знать устойчивые выражения с этим артиклем:
\begin{verbatim}
    in the box - в коробке
    It lies in the box.

    on the table/bed/sofa - на столе/кровати/диване
    It lies on the sofa.
\end{verbatim}

Для того, чтобы сказать, что одна вещь/человек принадлежит другому, в английском языке используют следующую
формулу (такие конструкции переводятся на русский язык начиная с конца):
\begin{verbatim}
    Обладатель + апостроф (') + вещь/человек
    He is my friend's trainer.
    He is my friend's teacher.
\end{verbatim}