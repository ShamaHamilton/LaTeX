\subsection{Урок №27}

\subsection*{\centering{Словарь}}
\begin{multicols}{2}
    \begin{enumerate}\setlength{\itemsep}{0pt}
        \item disagree - не соглашаться
        \item arrive - прибывать
        \item shopping - ходить по магазинам
        \item convenient - удобный
        \item basketball - баскетбол
        \item rent - арендовать, аренда
        \item funny - смешной, забавный
        \item hard - усердно, сложно
        \item sea - море
        \item higher - выше
        \item match - матч
        \item laugh - смеяться
        \item harder - усерднее, сложнее
        \item join - присоединиться
        \item holiday - отпуск
        \item solve - решать
        \item ocean - океан
        \item summer - лето
        \item volleyball - волейбол
    \end{enumerate}
\end{multicols}

\subsection*{\centering{Теория}}
Общая формула для образования вопросительных предложений в будущем времени:
\begin{verbatim}
    Will + I/we/you/they/he/she/it + глагол + ...
    Will he come to us?
    Will he learn English online?
\end{verbatim}

Нужно запомнить следующие устойчивые выражения:
\begin{verbatim}
    join us - присоединиться к нам (без предлога)
    Will they join us?

    go shopping - идти за покупками
    Will you go shopping now?
\end{verbatim}

Нужно запомнить следующие фразы с предлогами:
\begin{verbatim}
    go on holiday - поехать в отпуск (без артикля)
    Will you go on holiday this month?

    laugh at - смеяться над кем-то/чем-то
    Will he laugh at her?

    arrive in - прибывать в (страну, большой город)
    Will you arrive in Russia next week?
\end{verbatim}