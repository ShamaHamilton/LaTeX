\subsection{Урок №16}

\subsection*{\centering{Словарь}}
\begin{multicols}{2}
    \begin{enumerate}\setlength{\itemsep}{0pt}
        \item driver - водитель
        \item colleague - коллега
        \item assistant - помощник, ассистент
        \item cheaper - дешевле
        \item Brazil - Бразилия
        \item choice - выбор
        \item director - директор
        \item sit - сидеть
        \item kitchen - кухня
        \item trainer - тренер
        \item price - цена
        \item professional - профессиональный
        \item worker - работник
        \item native speaker - носитель языка
    \end{enumerate}
\end{multicols}

\subsection*{\centering{Теория}}
Общая формула образования вопросительных предложений в простом настоящем времени с to be (когда нет глагола):
\begin{verbatim}
    Am + I + ...?
    Am I right?

    Are + we/you/they + ...?
    Are they with you?

    Is + he/she/it + ...?
    Is it cheaper?
\end{verbatim}

Нужно запомнить степени сравнения прилагательного good (хороший):
\begin{verbatim}
    good - хороший
    Are they good doctors?

    better - лучше
    Is it better?

    the best - самый лучший
    Is it the best idea?
\end{verbatim}