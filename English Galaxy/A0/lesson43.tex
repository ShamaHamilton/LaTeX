\subsection{Урок №43}

\subsection*{\centering{Словарь}}
\begin{multicols}{2}
    \begin{enumerate}\setlength{\itemsep}{0pt}
        \item corner - угол
        \item incorrect - неправильный
        \item behind - за, сзади
        \item unusual - необычный
        \item classroom - классная комната
        \item yard - двор
    \end{enumerate}
\end{multicols}

\subsection*{\centering{Теория}}
Общая формула образования вопросительного предложения с вопросительными словами в прошедшем времени с глаголом
to be:
\begin{verbatim}
    Вопросительное слово + was + I/he/she/it + ...?
    Who was your teacher?
    Where were you yesterday?
\end{verbatim}