\subsection{Урок №13}

\subsection*{\centering{Словарь}}
\begin{multicols}{2}
    \begin{enumerate}\setlength{\itemsep}{0pt}
        \item at - в, на
        \item wonderful - замечательный, прекрасный
        \item best - лучший
        \item angry - злой, сердитый
        \item fact - факт
        \item husband - муж
        \item busy - занятый
        \item million - миллион
        \item sixty - шестьдесят
        \item right - правильный, правильно
        \item sure - уверен
        \item wife - жена
        \item old - старый
        \item singer - певец
        \item young - молодой
        \item secretary - секретарь
        \item room - комната
    \end{enumerate}
\end{multicols}

\subsection*{\centering{Теория}}
Если в предложении нет глагола в настоящем времени (в простом), то мы используем одну из форм глагола
to be - am/is/are:
\begin{verbatim}
    I + am + ...
    I am heppy

    We/you/they + are + ...
    You are absolutely right

    He/she/it + is + ...
    It is wonderful
\end{verbatim}