\subsection{Урок №17}

\subsection*{\centering{Словарь}}
\begin{multicols}{2}
    \begin{enumerate}\setlength{\itemsep}{0pt}
        \item worse - хуже
        \item wrong - неправильный
        \item trousers - брюки
        \item upset - расстроенный
        \item pair - пара
        \item box - коробка
        \item dress - платье
        \item bag - сумка
        \item jeans - джинсы
    \end{enumerate}
\end{multicols}

\subsection*{\centering{Теория}}
Общая формула для построения вопросительных предложений с вопросительными словами в простом настоящем
времени с глаголом to be:
\begin{verbatim}
    Вопросительное слово + are + we/you/they + ...?
    Why are you so busy?

    Вопросительное слово + is + he/she/it + ...?
    Why is he right?
\end{verbatim}

Нужно запомнить следующую вопросительную фразу:
\begin{verbatim}
    How old ...? - Сколько лет ...?
    How old + ...
    How old are they?
\end{verbatim}

В разговорной речи часто используют сокращенные версии вопросительных предложений:
\begin{verbatim}
    What's + ...
    What's in this box?
\end{verbatim}