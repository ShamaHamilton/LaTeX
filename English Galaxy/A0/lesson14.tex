\subsection{Урок №14}

\subsection*{\centering{Словарь}}
\begin{multicols}{2}
    \begin{enumerate}\setlength{\itemsep}{0pt}
        \item fine - хорошо, в порядке
        \item smart - умный
        \item difficult - сложный
        \item OK - окей, в порядке
        \item seventy - семьдесят
        \item single - холост
        \item married - женат, замужем
        \item possible - возможный
        \item rich - богатый
        \item ready - готов
        \item easy - легкий
        \item poor - бедный
        \item lazy - ленивый
        \item eighteen - восемнадцать
        \item beautiful - красивый
        \item dangerous - опасный
        \item hard-working - трудолюбивый
        \item eight - восемь
        \item parents - родители
        \item effective - эффективный
        \item alright - в порядке
    \end{enumerate}
\end{multicols}

\subsection*{\centering{Теория}}
В разговорной речи обычно используется краткая форма глагола to bo:
\begin{verbatim}
    He's/she's/it's + ...
    She's my mother

    I'm + ...
    I'm OK

    We're/you're/they're + ...
    We're at home
\end{verbatim}

В английском языке про возраст человека можно сказать только двумя способами (назвать цифру или же еще
добавить years old; если просто years, то это грамматически неверно):
\begin{verbatim}
    He's/she's + цифра + years old
    She's seventy years old.
    He's seventy.
\end{verbatim}