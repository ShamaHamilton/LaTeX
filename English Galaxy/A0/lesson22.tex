\subsection{Урок №22}

\subsection*{\centering{Словарь}}
\begin{multicols}{2}
    \begin{enumerate}\setlength{\itemsep}{0pt}
        \item afternoon - днем, после обеда
        \item o'clock - час, на часах
        \item stay - оставаться
        \item something - что-то
        \item should - следует
        \item ride - ездить, кататься
        \item please - пожалуйста
        \item check - проверять
        \item careful - осторожный
        \item ask - спрашивать, спросить
        \item continue - продолжать
        \item email - электронная почта
        \item horse - лошадь
        \item evening - вечер
        \item bike - велосипед
        \item must - должен
        \item meet - встречать
        \item information - информация
        \item lose - терять
        \item today - сегодня
        \item right now - прямо сейчас
        \item win - выиграть
        \item urgently - срочно
    \end{enumerate}
\end{multicols}

\subsection*{\centering{Теория}}
Нужно запомнить следующее устойчивое выражение:
\begin{verbatim}
    ride a bike - ездить на велосипеде
    ride + a + bike
    I can ride a bake.
\end{verbatim}

После глагола must (должен) следующий глагол ставится без каких-либо частиц, предлогов или окончаний:
\begin{verbatim}
    must + глагол
    I must do it.
    I must do it today.
\end{verbatim}