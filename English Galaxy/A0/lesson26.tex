\subsection{Урок №26}

\subsection*{\centering{Словарь}}
\begin{multicols}{2}
    \begin{enumerate}\setlength{\itemsep}{0pt}
        \item tonight - сегодня вечером
        \item Spain - Испания
        \item restaurant - ресторан
        \item sign - подписать
        \item supermarket - супермаркет
        \item successful - успешный
        \item party - вечеринка
        \item office - офис
        \item refuse - отказываться
        \item pass - сдавать
        \item immediately - немедленно
        \item mobile phone - мобильный телефон
    \end{enumerate}
\end{multicols}

\subsection*{\centering{Теория}}
Общая формула для образования отрицательных предложений в будущем времени:
\begin{verbatim}
    I/we/you/they/he/she/it + will + not + глагол + ...
    We will not live in that house.
\end{verbatim}

В разговорной речи обычно используется краткая форма:
\begin{verbatim}
    I/we/you/they/he/she/it + won't + глагол + ...
    I won't do those exercises.
    It won't be a mistake.
    It seems to me - мне кажется.
\end{verbatim}