\subsection{Урок №28}

\subsection*{\centering{Словарь}}
\begin{multicols}{2}
    \begin{enumerate}\setlength{\itemsep}{0pt}
        \item long - длинный, долго
        \item graduate - окончить
        \item whose - чей
        \item news - новости
        \item surprise - удивлять
        \item beach - пляж
        \item gym - зал, тренажерный зал
        \item Canada - Канада
        \item else - еще
        \item absent - отсутствовать
        \item last - длиться
        \item present - присутствующий
    \end{enumerate}
\end{multicols}

\subsection*{\centering{Теория}}
Общая формула для образования вопроса с вопросительными словами в будущем времени:
\begin{verbatim}
    Вопросительное слово/конструкция + will + I/we/you/they/he/she/it + ...?
    What will you do? - Что ты будешь делать?
    Why will it be necessary?
\end{verbatim}

Нужно запомнить следующие фразы с предлогами и устойчивые выражения:
\begin{verbatim}
    graduate from - закончить ВУЗ
    When will you graduate from university?

    get to - добраться куда-то
    When will she get to that place?

    get home - добраться до дома (без предлога)
\end{verbatim}