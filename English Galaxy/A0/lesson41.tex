\subsection{Урок №41}

\subsection*{\centering{Словарь}}
\begin{multicols}{2}
    \begin{enumerate}\setlength{\itemsep}{0pt}
        \item alone - одинокий
        \item pocket - карман
        \item ineffective - неэффективный
        \item spring - весна
        \item healthy - здоровый
        \item garden - сад
        \item autumn - осень
    \end{enumerate}
\end{multicols}

\subsection*{\centering{Теория}}
Общая формула образования отрицательных предложений в прошедшем времени с глаголом to be:
\begin{verbatim}
    I/he/she/it + wasn't + ...
    It wasn't a mistake.

    We/you/they + weren't + ...
    They weren't poor boys.
\end{verbatim}

Нужно запомнить:
\begin{verbatim}
    in winter/summer/spring/autumn - зимой/летом/весной/осенью
    I was there in winter.
\end{verbatim}