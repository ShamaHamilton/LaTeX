\subsection{Урок №4}

\subsection*{\centering{Словарь}}
\begin{multicols}{2}
    \begin{enumerate}\setlength{\itemsep}{0pt}
        \item meat - мясо
        \item son - сын
        \item cake - торт
        \item ice cream - мороженое
        \item chocolate - шоколад
        \item pizza - пицца
        \item swimming - плавание
        \item children - дети
        \item seafood - морепродукты
        \item child - ребёнок
        \item apartment - квартира
        \item airport - аэропорт
        \item daughter - дочь
        \item doyfriend - парень
        \item four - четыре
        \item German - немецкий
        \item foreign - иностранный
        \item girlfriend - девушка
        \item hospital - больница
        \item three - три
        \item Spanish - испанский
        \item women - женщины
        \item men - мужчины
    \end{enumerate}
\end{multicols}

\subsection*{\centering{Теория}}
Некоторые английские существительные образуют множественное число не по правилам, их нужно запомнить:
\begin{verbatim}
    man - men
    woman - women
    child - children
    We see two men.
    I see a man and two women.
    She has two children.
\end{verbatim}

Глагол have (иметь) в предложении после he / she / it превращается в has:
\begin{verbatim}
    he / she / it + has + ...
    He has a question.
    She has two daughters and a son.
\end{verbatim}

Сложные предложения в английском языке могут переводиться как с использованием that (что), так и без него:
\begin{verbatim}
    I + see
    I see you have a question.
    I + see + that
    I see that you have little free time.
\end{verbatim}