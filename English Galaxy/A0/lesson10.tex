\subsection{Урок №10}

\subsection*{\centering{Словарь}}
\begin{multicols}{2}
    \begin{enumerate}\setlength{\itemsep}{0pt}
        \item seven - семь
        \item than - чем
        \item thirty - тридцать
        \item seem - казаться
        \item look - смотреть
        \item important - важный
        \item method - метод
        \item listen - слушать
        \item strange - странный
        \item phone - телефон
        \item discuss - обсуждать
        \item from - из
        \item actress - актриса
        \item change - менять
        \item forty - сорок
        \item actor - актер
        \item one hundred - сто
        \item cheap - дешевый
        \item five - пять
        \item expensive - дорогой
        \item thousand - тысяча
        \item than - чем
    \end{enumerate}
\end{multicols}

\subsection*{\centering{Теория}}
Общая формула для построения вопросительных предложений в простом настоящем времени:
\begin{verbatim}
    Does + he/she/it + глагол + ...?
    Does he live in that house?
\end{verbatim}