\subsection{Урок №20}

\subsection*{\centering{Словарь}}
\begin{multicols}{2}
    \begin{enumerate}\setlength{\itemsep}{0pt}
        \item worried - взволнованный, обеспокоить
        \item wear - носить
        \item nowadays - в наше время
        \item useful - полезный
        \item walk - гулять
        \item choose - выбирать
        \item against - против
        \item formal - формальный
        \item informal - неформальный
        \item journalist - журналист
        \item casual - повседневный
    \end{enumerate}
\end{multicols}

\subsection*{\centering{Теория}}
Нужно запомнить следующие устойчивые выражения:
\begin{verbatim}
    much more - намного
    much + more
    Is it much more expensive?
\end{verbatim}

Общая формула образования вопросительных предложений в простом настоящем времени с to be (когда нет глагола):
\begin{verbatim}
    am I ...?
    Am I right?

    are we/you/they ...?
    Are they with you?

    is he/she/it ...?
    Is it cheaper?
\end{verbatim}

now - сейчас, это наречие времени, чаще всего ставится в конце предложения:
\begin{verbatim}
    I want to sleep now.
\end{verbatim}