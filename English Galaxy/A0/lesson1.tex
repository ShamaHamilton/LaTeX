\subsection{Урок №1}

\subsection*{\centering{Словарь}}
\begin{multicols}{2}
    \begin{enumerate}\setlength{\itemsep}{0pt}
        \item really - реально, действительно
        \item we - мы
        \item can - мочь, уметь
        \item it - это
        \item help - помогать, помочь
        \item me - меня, мне
        \item they - они
        \item learn - учиться, выучить
        \item you - ты, тебе
        \item work - работа, работать
        \item English - английский
        \item speak - говорить
        \item city - город
        \item online - онлайн
        \item Russia - Россия
        \item in - в
        \item my - мой
        \item want - хотеть
        \item like - нравиться
        \item Russian - русский
        \item I - Я
        \item study - учиться
        \item New York - Нью-Йорк
        \item place - место
        \item program - программа
        \item result - результат
        \item see - видеть
        \item undestand - понимать
        \item video - видео
        \item well - хорошо
        \item this - этот
        \item very - очень
        \item lesson - урок
        \item live - жить
        \item country - страна
        \item know - знать
        \item and - и
    \end{enumerate}
\end{multicols}

\subsection*{\centering{Теория}}
Общая форма для построения утвердительных предложений в простом настоящем времени в английском языке:
\begin{verbatim}
    Подлежащее + глагол(сказуемое) + ...
    I / We / You / They + глагол + ...
    I understand you.
    I travel every year.
\end{verbatim}

Некоторые слова всегда ставятся в одном и том же месте в предложении. Например: also (также, тоже) ставится между
подлежащим и глаголом:
\begin{verbatim}
    Подлежащее + also + глагол + ...
    I also think so.
\end{verbatim}

В английском языке есть очень много устойчивых выражений, которые надо запомнить:
\begin{verbatim}
    Speak English - говорить по-английски (без предлога).
    Go to work - ходить на работу.
\end{verbatim}

Перед словом like мы ставим именно личное местоимение:
\begin{verbatim}
    I / We / You / They + like + ...
    I like this language.
\end{verbatim}
Play this game - играть в эту игру.\\
Play football - играть в футбол.