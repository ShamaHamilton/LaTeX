\subsection{Урок №9}

\subsection*{\centering{Словарь}}
\begin{multicols}{2}
    \begin{enumerate}\setlength{\itemsep}{0pt}
        \item singing - пение
        \item weather - погода
        \item cinema - кино, кинотеатр
        \item show - показывать
        \item write - писать
        \item train - тренироваться, поезд
        \item support - поддержка, поддерживать
        \item reading - чтение
        \item real - реальный
        \item many - много
        \item park - парк
        \item dancing - танцы
        \item Chinese - китайский
        \item boring - скучный
        \item find - находить
        \item or - или
        \item intensively - интенсивно
        \item movie - фильм
    \end{enumerate}
\end{multicols}

\subsection*{\centering{Теория}}
Общая форма для построения вопросительных предложений в простом настоящем времени:
\begin{verbatim}
    Do + I/we/you/they + глагол + ...?
    Do you understand me?
    Do they buy it online?
\end{verbatim}

Если вопросительное предложение сложное (два подлежащих и два сказуемых), то вопросительный порядок
слов ставится только в первой части, во второй части слова стоят как в утвердительном предложении.
\begin{verbatim}
    Do you know + how + подлежащее + глагол
    Do you know how it works?

    Do you know + why + подлежащее + глагол
    Do you know why it happens?
\end{verbatim}

Нужно запомнить следующую конструкцию с like (нравится что-то делать):
\begin{verbatim}
    like + глагол+ing
    Do you like reading?
    Does she like studying?
\end{verbatim}