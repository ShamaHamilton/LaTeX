\subsection{Урок №40}

\subsection*{\centering{Словарь}}
\begin{multicols}{2}
    \begin{enumerate}\setlength{\itemsep}{0pt}
        \item terrible - ужасный
        \item tired - уставший
        \item worst - худший
        \item special - особенный, специальный
        \item village - деревня
        \item as - как
        \item guest - гость
        \item hungry - голодный
        \item ill - больной
        \item famous - известный
        \item normal - нормальный
        \item gift - подарок
        \item be out - быть не дома
        \item be away - отсутствовать
    \end{enumerate}
\end{multicols}

\subsection*{\centering{Теория}}
Общая формула образования утвердительных предложений в прошедшем времени с глаголом to be:
\begin{verbatim}
    we/you/they + were + ...
    You were right.

    I/he/she/it + was + ...
    It was good.
\end{verbatim}

Нужно запомнить:
\begin{verbatim}
    kind to - добр к кому-то
    kind + to + us/me/him/her/them
    She was so kind to us.

    in the afternoon - днем
    We were there in the afternoon.

    in the evening - вечером
    It was in the evening.

    in the morning - утром
    He was there in the morning.
\end{verbatim}