\subsection{Урок №21}

\subsection*{\centering{Словарь}}
\begin{multicols}{2}
    \begin{enumerate}\setlength{\itemsep}{0pt}
        \item window - окно
        \item smoke - курить
        \item table - стол
        \item swim - плавать
        \item could - мог, мог бы
        \item stop - останавливаться
        \item put - класть
        \item open - открывать
        \item story - история, рассказ
        \item door - дверь
        \item letter - письмо
        \item give - давать
        \item ofter - после
        \item example - пример
        \item floor - пол
        \item close - закрывать
        \item homework - домашнее задание
        \item later - позже
    \end{enumerate}
\end{multicols}

\subsection*{\centering{Теория}}
Предложения, состоящие только из просьбы или приказа (например, "Иди сюда!") переводятся на английский язык
при помощи глагола без каких-либо окончаний и частиц. В конце обычно ставится восклицательный знак:
\begin{verbatim}
    Глагол + ...!
    Say it!
    Close the door!
\end{verbatim}

После глаголов \underline{can} (уметь, мочь), \underline{could} (мог бы), \underline{would} (бы) следующий
глагол ставится без каких-либо частиц, предлогов и окончаний:
\begin{verbatim}
    can/could/would + глагол
    I can swim.
    I could buy this thing.
    I would buy this thing.
\end{verbatim}

Нужно запомнить следующее устойчивое выражение:
\begin{verbatim}
    I'd like to - я бы хотел
    I'd like to + глагол
    I'd like to order this food.
\end{verbatim}

Глаголы \underline{say} (сказать) и \underline{tell} (рассказывать) имеют схожие значения, но употребляются
по-разному: после tell обязательно должно стоять местоимение или существительное без предлога ("рассказать
кому-то"). Say употребляется само по себе.
\begin{verbatim}
    say + что-то
    You can say it now.

    tell + кому-то
    You can tell me it.
\end{verbatim}