\subsection{Урок №2}

\subsection*{\centering{Словарь}}
\begin{multicols}{2}
    \begin{enumerate}\setlength{\itemsep}{0pt}
        \item love - любовь, любить
        \item to - в, к
        \item there - там, туда
        \item about - о, об
        \item also - тоже, также
        \item more - больше, более
        \item here - здесь, сюда
        \item football - футбол
        \item book - книга
        \item go - идти, ехать
        \item so - так, такой
        \item read - читать
        \item school - школа
        \item song - песня
        \item buy - купить
        \item language - язык
        \item travelling - путешествие
        \item play - играть
        \item think - думать
        \item year - год
        \item music - музыка
        \item day - день
        \item game - игра
        \item often - часто
        \item do - делать
        \item practice - практика
        \item every - каждый
        \item travel - путешествовать
        \item usually - обычно
        \item sometimes - иногда
        \item these - эти
        \item thing - вещь
        \item absolutely - абсолютно, полностью
        \item ten - 10
        \item she - она
        \item twenty - 20
        \item far - далеко, далекий
        \item he - он
        \item near - рядом, около
        \item mistake - ошибка
        \item dollar - доллар
        \item ticket - билет
        \item cost - стоить
        \item feel - чувствовать себя
        \item together - вместе
        \item much - много
        \item better - лучше
        \item agree - соглашаться
        \item less - меньше
        \item fifty - 50
        \item euro - евро
        \item that - тот, что
        \item problem - проблема
        \item make - делать
        \item with - с
        \item pound - фунт
    \end{enumerate}
\end{multicols}

\subsection*{\centering{Теория}}
Общая форма для построения утвердительных предложений в простом настоящем времени в английском языке:
\begin{verbatim}
    Подлежащее + глагол(сказуемое) + ...
    He / She / It + глагол+s + ...
    He sees this problem.
    She understands it.
\end{verbatim}

В английском языке фразы со словом very(очень) ставятся только в конце предложения:
\begin{verbatim}
    Подлежащее + глагол + ... + very much.
    I like it very much.
    Подлежащее + глагол + ... + very far.
    She lives very far.
    He/She + like+s + ...
\end{verbatim}
Перед like мы ставим именно личное местоимение:
\begin{verbatim}
    She likes this game.
\end{verbatim}

Общее правило: множественное число существительных образуется добавлением окончания \underline{s}:
\begin{verbatim}
    Существительное + s.
    It costs twenty euros.
    She makes mistakes.
\end{verbatim}

Наречия often (часто), usually (обычно) ставятся между подлежащим и глаголом (сказуемым):
\begin{verbatim}
    Подлежащее + often/usually + глагол + ...
\end{verbatim}