\subsection{Урок №12}

\subsection*{\centering{Словарь}}
\begin{multicols}{2}
    \begin{enumerate}\setlength{\itemsep}{0pt}
        \item unhappy - несчастный
        \item sad - печальный
        \item which - который
        \item when - когда
        \item what - что, какие
        \item leave - покидать, уезжать
        \item where - где
        \item download - скачать
        \item all - все
        \item file - файл
        \item level - уровень
        \item regularly - регулярно
        \item document - документ
        \item island - остров
    \end{enumerate}
\end{multicols}

\subsection*{\centering{Теория}}
Общая формула для построения вопросительных предложений с вопросительными словами:
\begin{verbatim}
    Вопросительное слово + do + I/we/you/they + глагол + ...?
    Why do you read it?

    Вопросительное слово + does + he/she/it + глагол + ...?
    When does it happen?

    Вопросительные слова и фразы:
    When/Where/What/Why + ...
    When do you get up?
    Where does she study?
    What do you prefer?

    Why/How/How much/How often/What time + ...
    Why do you do it?
    How do you feel?
    How much does it cost?
    How often do you speak English?
    What time do you come?
\end{verbatim}