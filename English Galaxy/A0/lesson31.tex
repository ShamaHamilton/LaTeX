\subsection{Урок №31}

\subsection*{\centering{Словарь}}
\begin{multicols}{2}
    \begin{enumerate}\setlength{\itemsep}{0pt}
        \item wednesday - среда
        \item word - слово
        \item sunday - воскресенье
        \item thursday - четверг
        \item badminton - бадминтон
        \item sound - звук, звучать
        \item country house - загородный дом
        \item tuesday - вторник
        \item anything - что-либо
        \item explain - объяснять
        \item physical - физический
        \item friday - пятница
        \item night - ночь
        \item saturday - суббота
        \item lock - запирать
        \item monday - понедельник
        \item forget - забывать
        \item past - прошлое
    \end{enumerate}
\end{multicols}

\subsection*{\centering{Теория}}
Общая форма образования отрицательных предложений в прошедшем времени:
\begin{verbatim}
    didn't + глагол
    He didn't have money.
    I didn't close the door.
\end{verbatim}

Нужно запомнить следующие фразы:
\begin{verbatim}
    do exercises - делать упражнения
    on Sunday - в воскресенье (можно заменить на любой другой день недели)
    do + these physical + exercises
    I didn't do these physical exercises.
    I didn't go to my country house on Sunday.
\end{verbatim}