\subsection{Урок №33}

\subsection*{\centering{Словарь}}
\begin{multicols}{2}
    \begin{enumerate}\setlength{\itemsep}{0pt}
        \item winter - зима
        \item sweater - свитер
        \item decide - решать
        \item earlier - раньше
    \end{enumerate}
\end{multicols}

\subsection*{\centering{Теория}}
Общая формула образования вопроса в прошедшем времени:
\begin{verbatim}
    Вопросительное слово + did + I/we/you/they/he/she/it + глагол в первой
форме + ...?
    What did you see there?
    How did you learn English?
\end{verbatim}

Нужно запомнить:
\begin{verbatim}
    return home - вернуться домой (без предлога)
    When did you return home?

    for dinner/breakfast/lunch - на ужин/завтрак/обед
    What did you eat for dinner.
\end{verbatim}