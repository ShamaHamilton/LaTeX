\subsection{Урок №7}

\subsection*{\centering{Словарь}}
\begin{multicols}{2}
    \begin{enumerate}\setlength{\itemsep}{0pt}
        \item internet - интернет
        \item test - тест
        \item teach - учить, преподавать
        \item sell - продавать
        \item exam - экзамен
        \item worry - волноваться, беспокоиться
        \item motivate - мотивировать
        \item cat - кошка, кот
        \item maths - математика
        \item plan - план
        \item friend - друг
        \item use - использовать
        \item dog - собака
        \item interest - интересовать
        \item motivate - мотивировать
        \item service - сервис, услуга
        \item only - только
        \item plan - план
        \item sell - продавать
        \item interesting - интересный
        \item any - какой-либо
        \item subject - предмет, тема
        \item use - использовать
    \end{enumerate}
\end{multicols}

\subsection*{\centering{Теория}}
Общая формула для построения отрицательных предложений в простом настоящем времени:
\begin{verbatim}
    He / She / It + doesn't + глагол + ...
    It doesn't interest them.
    He doesn't like it.
\end{verbatim}

Слово any (какой-либо) употребляется только в отрицательных или вопросительных предложениях, когда
существительное во множественном числе или неисчисляемое:
\begin{verbatim}
    any + questions/plans ...
    He doesn't have any questions.
    I don't have any plans.
\end{verbatim}