\subsection{Урок №30}

\subsection*{\centering{Словарь}}
\begin{multicols}{2}
    \begin{enumerate}\setlength{\itemsep}{0pt}
        \item app - приложение
        \item fail - провалить
        \item compare - сравнивать
        \item jump - прыгать
        \item France - Франция
        \item link - ссылка
        \item high - высокий, высоко
        \item ago - назад
        \item move - двигаться
        \item quickly - быстро
        \item training - тренировка
        \item yesterday - вчера
        \item correct - исправлять
        \item save - сохранять
        \item trip - поездка
        \item share - делиться
    \end{enumerate}
\end{multicols}

\subsection*{\centering{Теория}}
Общая формула для образования прошедшего времени с правильными глаголами:
\begin{verbatim}
    глагол+ed
    They moved very quickly.
    She asked me a question.
\end{verbatim}

У неправильных глаголов нет общего правила образования прошедшего времени. Их вторую форму нужно запомнить:
\begin{verbatim}
    go - went - gone
    go - went
    He went to school

    do - did - done

    make - made
    He made some mistakes.
\end{verbatim}

Нужно запомнить следующие устойчивые выражения:
\begin{verbatim}
    last week/month/year (без предлога)
    I visited France last year.
\end{verbatim}